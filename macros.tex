\documentclass{myassignment}

\title{Common mathematical notations}
\author{}
\date{}

\begin{document}

\maketitle

\addtocounter{section}{1}

\textbf{Fonts in math mode}
\begin{align*}
    &\text{\texttt{\textbackslash none}: } &&abcdefghijklmnopqrstuvwxyz \\
    &\text{\texttt{\textbackslash mathrm}: } &&\mathrm{abcdefghijklmnopqrstuvwxyz} \\
    &\text{\texttt{\textbackslash mathbf}: } &&\mathbf{abcdefghijklmnopqrstuvwxyz} \\
    &\text{\texttt{\textbackslash boldsymbol}: } &&\boldsymbol{abcdefghijklmnopqrstuvwxyz} \\
    &\text{\texttt{\textbackslash mathfrak}: } &&\mathfrak{abcdefghijklmnopqrstuvwxyz} \\
    \\
    &\text{\texttt{\textbackslash none}: } &&ABCDEFGHIJKLMNOPQRSTUVWXYZ \\
    &\text{\texttt{\textbackslash mathrm}: } &&\mathrm{ABCDEFGHIJKLMNOPQRSTUVWXYZ} \\
    &\text{\texttt{\textbackslash mathbf}: } &&\mathbf{ABCDEFGHIJKLMNOPQRSTUVWXYZ} \\
    &\text{\texttt{\textbackslash boldsymbol}: } &&\boldsymbol{ABCDEFGHIJKLMNOPQRSTUVWXYZ} \\
    &\text{\texttt{\textbackslash mathbb}: } &&\mathbb{ABCDEFGHIJKLMNOPQRSTUVWXYZ} \\
    &\text{\texttt{\textbackslash mathcal}: } &&\mathcal{ABCDEFGHIJKLMNOPQRSTUVWXYZ} \\
    &\text{\texttt{\textbackslash mathfrak}: } &&\mathfrak{ABCDEFGHIJKLMNOPQRSTUVWXYZ}
\end{align*}

\textbf{Letter modifiers}
\begin{align*}
    &\text{\texttt{\textbackslash tilde}: } &&\tilde{a}\,\tilde{b}\,\tilde{c}\,\tilde{d}\,\tilde{e}\,\tilde{f}\,\tilde{g}\,\tilde{h}\,\tilde{i}\,\tilde{j}\,\tilde{k}\,\tilde{l}\,\tilde{m}\,\tilde{n}\,\tilde{o}\,\tilde{p}\,\tilde{q}\,\tilde{r}\,\tilde{s}\,\tilde{t}\,\tilde{u}\,\tilde{v}\,\tilde{w}\,\tilde{x}\,\tilde{y}\,\tilde{z} \\
    &\text{\texttt{\textbackslash narrowtilde}: } &&\narrowtilde{a}\,\narrowtilde{b}\,\narrowtilde{c}\,\narrowtilde{d}\,\narrowtilde{e}\,\narrowtilde{f}\,\narrowtilde{g}\,\narrowtilde{h}\,\narrowtilde{i}\,\narrowtilde{j}\,\narrowtilde{k}\,\narrowtilde{l}\,\narrowtilde{m}\,\narrowtilde{n}\,\narrowtilde{o}\,\narrowtilde{p}\,\narrowtilde{q}\,\narrowtilde{r}\,\narrowtilde{s}\,\narrowtilde{t}\,\narrowtilde{u}\,\narrowtilde{v}\,\narrowtilde{w}\,\narrowtilde{x}\,\narrowtilde{y}\,\narrowtilde{z} \\
    &\text{\texttt{\textbackslash hat}: } &&\hat{a}\,\hat{b}\,\hat{c}\,\hat{d}\,\hat{e}\,\hat{f}\,\hat{g}\,\hat{h}\,\hat{i}\,\hat{j}\,\hat{k}\,\hat{l}\,\hat{m}\,\hat{n}\,\hat{o}\,\hat{p}\,\hat{q}\,\hat{r}\,\hat{s}\,\hat{t}\,\hat{u}\,\hat{v}\,\hat{w}\,\hat{x}\,\hat{y}\,\hat{z} \\
    &\text{\texttt{\textbackslash narrowhat}: } &&\narrowhat{a}\,\narrowhat{b}\,\narrowhat{c}\,\narrowhat{d}\,\narrowhat{e}\,\narrowhat{f}\,\narrowhat{g}\,\narrowhat{h}\,\narrowhat{i}\,\narrowhat{j}\,\narrowhat{k}\,\narrowhat{l}\,\narrowhat{m}\,\narrowhat{n}\,\narrowhat{o}\,\narrowhat{p}\,\narrowhat{q}\,\narrowhat{r}\,\narrowhat{s}\,\narrowhat{t}\,\narrowhat{u}\,\narrowhat{v}\,\narrowhat{w}\,\narrowhat{x}\,\narrowhat{y}\,\narrowhat{z} \\
    &\text{\texttt{\textbackslash dot}: } &&\dot{a}\,\dot{b}\,\dot{c}\,\dot{d}\,\dot{e}\,\dot{f}\,\dot{g}\,\dot{h}\,\dot{i}\,\dot{j}\,\dot{k}\,\dot{l}\,\dot{m}\,\dot{n}\,\dot{o}\,\dot{p}\,\dot{q}\,\dot{r}\,\dot{s}\,\dot{t}\,\dot{u}\,\dot{v}\,\dot{w}\,\dot{x}\,\dot{y}\,\dot{z} \\
    &\text{\texttt{\textbackslash ddot}: } &&\ddot{a}\,\ddot{b}\,\ddot{c}\,\ddot{d}\,\ddot{e}\,\ddot{f}\,\ddot{g}\,\ddot{h}\,\ddot{i}\,\ddot{j}\,\ddot{k}\,\ddot{l}\,\ddot{m}\,\ddot{n}\,\ddot{o}\,\ddot{p}\,\ddot{q}\,\ddot{r}\,\ddot{s}\,\ddot{t}\,\ddot{u}\,\ddot{v}\,\ddot{w}\,\ddot{x}\,\ddot{y}\,\ddot{z}
\end{align*}

\textbf{Operators}

Here are some commonly used operators, which can be accessed with \texttt{\textbackslash opn}.

\begin{align*}
    &\sin & &\cos & &\tan & &\cot \\
    &\arcsin & &\arccos & &\arctan \\
    &\sinh & &\cosh && \tanh & &\coth \\
    &\sin^{-1} & &\cos^{-1} & &\tan^{-1} & &\cot^{-1}
\end{align*}

\begin{align*}
    &\exp & &\log & &\ln & &\lg
\end{align*}

Groups

Linear algebra

Matrices

Number theory

\textbf{Coding theory}

\begin{align*}
    \wt(x) && \ev_\mathcal{P}(f)
\end{align*}

\textbf{Differentials}

The \texttt{$\backslash$dd} command should have the right spacing.

\begin{align*}
    a\dd x + b\dd y && \int_0^\infty \frac{\sin x}{x} \dd x && \int_0^\pi \sin x \dd x
\end{align*}

\textbf{Integrals}

Integrals can be typeset with.

\begin{align*}
    \int_a^b \sin x \dd x && \int_{\frac{\pi}{2}}^{\sqrt{\pi^2 - 1}} x^2 + e^{\cos x} \dd x
\end{align*}

\textbf{Complex analysis}

\begin{align*}
    \Re(z) && \Im(z)
\end{align*}

Probability theory

Figures

Commutative diagrams

\textbf{Arrows}

\begin{align*}
    f \colon A \to B && A \into B && A \stackrel{f}{\onto} B
\end{align*}

\textbf{Set definitions}

\begin{align*}
    A &= \{ x \in X \mid p(x) \} \\
    B &= \{ y \in Y \colon q(y) \}
\end{align*}


\textbf{Enumerate}

We can create an ordered list.

\begin{enumerate}[i.]
    \item First item
    \item Second item
    \begin{enumerate}[(a)]
        \item First subitem
    \end{enumerate}
    \item Third item
\end{enumerate}

We can also include some text in the middle and resume with the list.

\begin{enumerate}[resume*]
    \item Fourth item
    \item Fifth item
\end{enumerate}

Similary, we can create an unordered list.

\begin{itemize}
    \item An item
    \item Another item
\end{itemize}

\textbf{Intelligent comma}

The spacing is correct when using a comma as a decimal separator, but also when using the comma as a separator normally when including a space.
\begin{align*}
    3,1415926535 && (1, 2)
\end{align*}

\textbf{Theorem environments}

\begin{theorem}\label{thm:first}
Here is a theorem.
\end{theorem}

% Here you need to put the extra {} so that the [] do not confuse the compiler
\begin{lemma}[Euler {\cite[page 3]{euler1785}}]\label{lem:second}
Here is a named lemma.
\end{lemma}

\begin{proof}
This is the proof of the above lemma.
\end{proof}

\begin{proof}[Proof of Theorem \ref{thm:first}]
This is the proof for the above theorem.
\end{proof}

Display environments

\bibliographystyle{ieeetr}
\bibliography{bib.bib}

\end{document}