\documentclass{myassignment}

\title{MyMacros package documentation}
\author{Okko Makkonen}
\date{February 11, 2023}

\begin{document}

\maketitle

\addtocounter{section}{1}

\textbf{Fonts in math mode}
\begin{align*}
    &\text{normal: } &&abcdefghijklmnopqrstuvwxyz \\
    &\text{\texttt{\textbackslash mathrm}: } &&\mathrm{abcdefghijklmnopqrstuvwxyz} \\
    &\text{\texttt{\textbackslash mathbf}: } &&\mathbf{abcdefghijklmnopqrstuvwxyz} \\
    &\text{\texttt{\textbackslash boldsymbol}: } &&\boldsymbol{abcdefghijklmnopqrstuvwxyz} \\
    &\text{\texttt{\textbackslash mathfrak}: } &&\mathfrak{abcdefghijklmnopqrstuvwxyz} \\
    \\
    &\text{normal: } &&ABCDEFGHIJKLMNOPQRSTUVWXYZ \\
    &\text{\texttt{\textbackslash mathrm}: } &&\mathrm{ABCDEFGHIJKLMNOPQRSTUVWXYZ} \\
    &\text{\texttt{\textbackslash mathbf}: } &&\mathbf{ABCDEFGHIJKLMNOPQRSTUVWXYZ} \\
    &\text{\texttt{\textbackslash boldsymbol}: } &&\boldsymbol{ABCDEFGHIJKLMNOPQRSTUVWXYZ} \\
    &\text{\texttt{\textbackslash mathbb}: } &&\mathbb{ABCDEFGHIJKLMNOPQRSTUVWXYZ} \\
    &\text{\texttt{\textbackslash mathcal}: } &&\mathcal{ABCDEFGHIJKLMNOPQRSTUVWXYZ} \\
    &\text{\texttt{\textbackslash mathfrak}: } &&\mathfrak{ABCDEFGHIJKLMNOPQRSTUVWXYZ} \\
    &\text{\texttt{\textbackslash mathscr}: } &&\mathscr{ABCDEFGHIJKLMNOPQRSTUVWXYZ}
\end{align*}

\textbf{Letter modifiers}
\begin{align*}
    &\text{\texttt{\textbackslash bar}: } &&\bar{a}\,\,\bar{b}\,\,\bar{c}\,\,\bar{d}\,\,\bar{e}\,\,\bar{f}\,\,\bar{g}\,\,\bar{h}\,\,\bar{i}\,\,\bar{j}\,\,\bar{k}\,\,\bar{l}\,\,\bar{m}\,\,\bar{n}\,\,\bar{o}\,\,\bar{p}\,\,\bar{q}\,\,\bar{r}\,\,\bar{s}\,\,\bar{t}\,\,\bar{u}\,\,\bar{v}\,\,\bar{w}\,\,\bar{x}\,\,\bar{y}\,\,\bar{z} \\
    &&&\bar{A}\,\bar{B}\,\bar{C}\,\bar{D}\,\bar{E}\,\bar{F}\,\bar{G}\,\bar{H}\,\bar{I}\,\bar{J}\,\bar{K}\,\bar{L}\,\bar{M}\,\bar{N}\,\bar{O}\,\bar{P}\,\bar{Q}\,\bar{R}\,\bar{S}\,\bar{T}\,\bar{U}\,\bar{V}\,\bar{W}\,\bar{X}\,\bar{Y}\,\bar{Z} \\
    &\text{\texttt{\textbackslash overline}: } &&\overline{a}\,\,\overline{b}\,\,\overline{c}\,\,\overline{d}\,\,\overline{e}\,\,\overline{f}\,\,\overline{g}\,\,\overline{h}\,\,\overline{i}\,\,\overline{j}\,\,\overline{k}\,\,\overline{l}\,\,\overline{m}\,\,\overline{n}\,\,\overline{o}\,\,\overline{p}\,\,\overline{q}\,\,\overline{r}\,\,\overline{s}\,\,\overline{t}\,\,\overline{u}\,\,\overline{v}\,\,\overline{w}\,\,\overline{x}\,\,\overline{y}\,\,\overline{z} \\
    &&&\overline{A}\,\overline{B}\,\overline{C}\,\overline{D}\,\overline{E}\,\overline{F}\,\overline{G}\,\overline{H}\,\overline{I}\,\overline{J}\,\overline{K}\,\overline{L}\,\overline{M}\,\overline{N}\,\overline{O}\,\overline{P}\,\overline{Q}\,\overline{R}\,\overline{S}\,\overline{T}\,\overline{U}\,\overline{V}\,\overline{W}\,\overline{X}\,\overline{Y}\,\overline{Z} \\
    &\text{\texttt{\textbackslash tilde}: } &&\tilde{a}\,\,\tilde{b}\,\,\tilde{c}\,\,\tilde{d}\,\,\tilde{e}\,\,\tilde{f}\,\,\tilde{g}\,\,\tilde{h}\,\,\tilde{i}\,\,\tilde{j}\,\,\tilde{k}\,\,\tilde{l}\,\,\tilde{m}\,\,\tilde{n}\,\,\tilde{o}\,\,\tilde{p}\,\,\tilde{q}\,\,\tilde{r}\,\,\tilde{s}\,\,\tilde{t}\,\,\tilde{u}\,\,\tilde{v}\,\,\tilde{w}\,\,\tilde{x}\,\,\tilde{y}\,\,\tilde{z} \\
    &&&\tilde{A}\,\tilde{B}\,\tilde{C}\,\tilde{D}\,\tilde{E}\,\tilde{F}\,\tilde{G}\,\tilde{H}\,\tilde{I}\,\tilde{J}\,\tilde{K}\,\tilde{L}\,\tilde{M}\,\tilde{N}\,\tilde{O}\,\tilde{P}\,\tilde{Q}\,\tilde{R}\,\tilde{S}\,\tilde{T}\,\tilde{U}\,\tilde{V}\,\tilde{W}\,\tilde{X}\,\tilde{Y}\,\tilde{Z} \\
    &\text{\texttt{\textbackslash narrowtilde}: } &&\narrowtilde{a}\,\,\narrowtilde{b}\,\,\narrowtilde{c}\,\,\narrowtilde{d}\,\,\narrowtilde{e}\,\,\narrowtilde{f}\,\,\narrowtilde{g}\,\,\narrowtilde{h}\,\,\narrowtilde{i}\,\,\narrowtilde{j}\,\,\narrowtilde{k}\,\,\narrowtilde{l}\,\,\narrowtilde{m}\,\,\narrowtilde{n}\,\,\narrowtilde{o}\,\,\narrowtilde{p}\,\,\narrowtilde{q}\,\,\narrowtilde{r}\,\,\narrowtilde{s}\,\,\narrowtilde{t}\,\,\narrowtilde{u}\,\,\narrowtilde{v}\,\,\narrowtilde{w}\,\,\narrowtilde{x}\,\,\narrowtilde{y}\,\,\narrowtilde{z} \\
    &&&\narrowtilde{A}\,\narrowtilde{B}\,\narrowtilde{C}\,\narrowtilde{D}\,\narrowtilde{E}\,\narrowtilde{F}\,\narrowtilde{G}\,\narrowtilde{H}\,\narrowtilde{I}\,\narrowtilde{J}\,\narrowtilde{K}\,\narrowtilde{L}\,\narrowtilde{M}\,\narrowtilde{N}\,\narrowtilde{O}\,\narrowtilde{P}\,\narrowtilde{Q}\,\narrowtilde{R}\,\narrowtilde{S}\,\narrowtilde{T}\,\narrowtilde{U}\,\narrowtilde{V}\,\narrowtilde{W}\,\narrowtilde{X}\,\narrowtilde{Y}\,\narrowtilde{Z} \\
    &\text{\texttt{\textbackslash hat}: } &&\hat{a}\,\,\hat{b}\,\,\hat{c}\,\,\hat{d}\,\,\hat{e}\,\,\hat{f}\,\,\hat{g}\,\,\hat{h}\,\,\hat{i}\,\,\hat{j}\,\,\hat{k}\,\,\hat{l}\,\,\hat{m}\,\,\hat{n}\,\,\hat{o}\,\,\hat{p}\,\,\hat{q}\,\,\hat{r}\,\,\hat{s}\,\,\hat{t}\,\,\hat{u}\,\,\hat{v}\,\,\hat{w}\,\,\hat{x}\,\,\hat{y}\,\,\hat{z} \\
    &&&\hat{A}\,\hat{B}\,\hat{C}\,\hat{D}\,\hat{E}\,\hat{F}\,\hat{G}\,\hat{H}\,\hat{I}\,\hat{J}\,\hat{K}\,\hat{L}\,\hat{M}\,\hat{N}\,\hat{O}\,\hat{P}\,\hat{Q}\,\hat{R}\,\hat{S}\,\hat{T}\,\hat{U}\,\hat{V}\,\hat{W}\,\hat{X}\,\hat{Y}\,\hat{Z} \\
    &\text{\texttt{\textbackslash narrowhat}: } &&\narrowhat{a}\,\,\narrowhat{b}\,\,\narrowhat{c}\,\,\narrowhat{d}\,\,\narrowhat{e}\,\,\narrowhat{f}\,\,\narrowhat{g}\,\,\narrowhat{h}\,\,\narrowhat{i}\,\,\narrowhat{j}\,\,\narrowhat{k}\,\,\narrowhat{l}\,\,\narrowhat{m}\,\,\narrowhat{n}\,\,\narrowhat{o}\,\,\narrowhat{p}\,\,\narrowhat{q}\,\,\narrowhat{r}\,\,\narrowhat{s}\,\,\narrowhat{t}\,\,\narrowhat{u}\,\,\narrowhat{v}\,\,\narrowhat{w}\,\,\narrowhat{x}\,\,\narrowhat{y}\,\,\narrowhat{z} \\
    &&&\narrowhat{A}\,\narrowhat{B}\,\narrowhat{C}\,\narrowhat{D}\,\narrowhat{E}\,\narrowhat{F}\,\narrowhat{G}\,\narrowhat{H}\,\narrowhat{I}\,\narrowhat{J}\,\narrowhat{K}\,\narrowhat{L}\,\narrowhat{M}\,\narrowhat{N}\,\narrowhat{O}\,\narrowhat{P}\,\narrowhat{Q}\,\narrowhat{R}\,\narrowhat{S}\,\narrowhat{T}\,\narrowhat{U}\,\narrowhat{V}\,\narrowhat{W}\,\narrowhat{X}\,\narrowhat{Y}\,\narrowhat{Z} \\
    &\text{\texttt{\textbackslash dot}: } &&\dot{a}\,\,\dot{b}\,\,\dot{c}\,\,\dot{d}\,\,\dot{e}\,\,\dot{f}\,\,\dot{g}\,\,\dot{h}\,\,\dot{i}\,\,\dot{j}\,\,\dot{k}\,\,\dot{l}\,\,\dot{m}\,\,\dot{n}\,\,\dot{o}\,\,\dot{p}\,\,\dot{q}\,\,\dot{r}\,\,\dot{s}\,\,\dot{t}\,\,\dot{u}\,\,\dot{v}\,\,\dot{w}\,\,\dot{x}\,\,\dot{y}\,\,\dot{z} \\
    &&&\dot{A}\,\dot{B}\,\dot{C}\,\dot{D}\,\dot{E}\,\dot{F}\,\dot{G}\,\dot{H}\,\dot{I}\,\dot{J}\,\dot{K}\,\dot{L}\,\dot{M}\,\dot{N}\,\dot{O}\,\dot{P}\,\dot{Q}\,\dot{R}\,\dot{S}\,\dot{T}\,\dot{U}\,\dot{V}\,\dot{W}\,\dot{X}\,\dot{Y}\,\dot{Z} \\
    &\text{\texttt{\textbackslash ddot}: } &&\ddot{a}\,\,\ddot{b}\,\,\ddot{c}\,\,\ddot{d}\,\,\ddot{e}\,\,\ddot{f}\,\,\ddot{g}\,\,\ddot{h}\,\,\ddot{i}\,\,\ddot{j}\,\,\ddot{k}\,\,\ddot{l}\,\,\ddot{m}\,\,\ddot{n}\,\,\ddot{o}\,\,\ddot{p}\,\,\ddot{q}\,\,\ddot{r}\,\,\ddot{s}\,\,\ddot{t}\,\,\ddot{u}\,\,\ddot{v}\,\,\ddot{w}\,\,\ddot{x}\,\,\ddot{y}\,\,\ddot{z} \\
    &&&\ddot{A}\,\ddot{B}\,\ddot{C}\,\ddot{D}\,\ddot{E}\,\ddot{F}\,\ddot{G}\,\ddot{H}\,\ddot{I}\,\ddot{J}\,\ddot{K}\,\ddot{L}\,\ddot{M}\,\ddot{N}\,\ddot{O}\,\ddot{P}\,\ddot{Q}\,\ddot{R}\,\ddot{S}\,\ddot{T}\,\ddot{U}\,\ddot{V}\,\ddot{W}\,\ddot{X}\,\ddot{Y}\,\ddot{Z}
\end{align*}

\newpage

\textbf{Common notation}

Differentials can be written with \texttt{$\backslash$dd}.
\begin{align*}
    a\dd x + b\dd y && \int_0^\infty \frac{\sin x}{x} \dd x && \int_{\R^n} f(x) \dd \mu(x)
\end{align*}

Integrals can be typeset with \texttt{\textbackslash int}, \texttt{\textbackslash iint}, \texttt{\textbackslash oint}, \texttt{\textbackslash dint}.
\begin{align*}
    \int_a^b \sin x \dd x && \iint_A f(x, y) \dd \lambda(x, y) && \oint_\gamma \ln z \dd z && \dint_Q f(x) \dd x
\end{align*}

The commands \texttt{\textbackslash Re} and \texttt{\textbackslash Im} have been redefined.
\begin{align*}
    \Re(z) && \Im(z)
\end{align*}

For probability theory we have \texttt{\textbackslash Pr}, \texttt{\textbackslash E} and \texttt{\textbackslash Var}.
\begin{align*}
    \Pr{X \in A} && \E{X^2} && \Var{X}
\end{align*}

For common arrows we have \texttt{\textbackslash to}, \texttt{\textbackslash into} and \texttt{\textbackslash onto}. For setting symbols above and below other symbols use \texttt{\textbackslash overset} and \texttt{\textbackslash underset}.
\begin{align*}
    f \colon A \to B && A \into B && A \overset{f}{\onto} B
\end{align*}

Multiline quantifiers can be written with \texttt{\textbackslash substack}.
\begin{align*}
    \sum_{\substack{i \in \Z \\ i~\text{odd}}} \frac{1}{i^2} = \frac{\pi^2}{4} && p(x, y) = \sum_{\substack{i, j \in \Z \\ i, j \geq 0 \\ i + j \leq 100}} x^i y^j
\end{align*}

Use \texttt{\textbackslash loc} to denote local spaces: $L_\loc^1(\R^n)$.

The following commands use the variant version, \texttt{\textbackslash epsilon}, \texttt{\textbackslash phi}, \texttt{\textbackslash emptyset}, \texttt{\textbackslash leq} and \texttt{\textbackslash geq}.
\begin{align*}
    \epsilon && \phi && \emptyset && \leq && \geq
\end{align*}

The old symbols can still be accessed with \texttt{\textbackslash le} and \texttt{\textbackslash ge}: $\le$ and $\ge$.

The following \texttt{\textbackslash mathbb} variables can be accessed with \texttt{\textbackslash N}, \texttt{\textbackslash Z}, \texttt{\textbackslash Q}, \texttt{\textbackslash R}, \texttt{\textbackslash C}, \texttt{\textbackslash F}, \texttt{\textbackslash K}, \texttt{\textbackslash P}, \texttt{\textbackslash V} and \texttt{\textbackslash I}.
\begin{align*}
    \N && \Z && \Q && \R && \C && \F && \K && \P && \V && \I
\end{align*}
Additionally, \texttt{\textbackslash 1} can be used to write $\1$. The old \texttt{\textbackslash P} can still be accessed with \texttt{\textbackslash pilcrow}: \pilcrow.

The \texttt{\textbackslash prep} can be used to write the \texttt{\textbackslash perp} before the variable: $\prep V$.

\textbf{Latin abbreviations}

The Latin abbreviations can be written with \texttt{\textbackslash ie}, \texttt{\textbackslash eg}, \texttt{\textbackslash etal} and \texttt{\textbackslash etc}: \ie, \eg, \etal, \etc.

\newpage

\textbf{Enumerate}

We can create an ordered list.

\begin{enumerate}[i.]
    \item First item
    \item Second item
    \begin{enumerate}[(a)]
        \item First subitem
    \end{enumerate}
    \item Third item
\end{enumerate}

We can also include some text in the middle and resume with the list.

\begin{enumerate}[resume*]
    \item Fourth item
    \item Fifth item
\end{enumerate}

Similary, we can create an unordered list.

\begin{itemize}
    \item An item
    \item Another item
\end{itemize}

\textbf{Fixes}

The spacing is correct when using a comma as a decimal separator, but also when using the comma as a separator normally when including a space.
\begin{align*}
    \pi = 3,1415926535\dots && (1, 2)
\end{align*}

The spacing of delimiters is fixed, \ie, it is safe to use \texttt{\textbackslash left} and \texttt{\textbackslash right}.
\begin{align*}
    \sin \left(\frac{1}{2}\right)
\end{align*}

\textbf{Theorem environments}

\begin{theorem}\label{thm:first}
Let $R$ be a ring. If $A, B \in R$ are such that $AB = BA$, then
\begin{align*}
    (A + B)(A - B) = A^2 - B^2.
\end{align*}
\end{theorem}

% Here you need to put the extra {} so that the [] do not confuse the compiler
\begin{lemma}[Euclid {\cite[page 3]{euclid400BCE}}]\label{lem:second}
Here is a named lemma.
\end{lemma}

\begin{proof}
This is the proof of the above lemma.

\BoxedRightarrow Denote this direction with \texttt{\textbackslash BoxedRightarrow}.

\BoxedLeftarrow Denote this direction with \texttt{\textbackslash BoxedLeftarrow}.
\end{proof}

\begin{proof}[Proof of Theorem \ref{thm:first}]
This is the proof for the above theorem.
\begin{align*}
    (A + B)(A - B) &= AA - AB + BA - BB \\
    \annotate{By commutativity of $A$ and $B$}
    &= AA - AB + AB - BB \\
    \annotate{By canceling the terms}
    &= A^2 - B^2
\end{align*}
\end{proof}

\bibliographystyle{ieeetr}
\bibliography{bib.bib}

\end{document}