% The myassignment class can be found in https://github.com/okkomakkonen/latex-macros
\documentclass{myassignment}

\title{Sample Assignment 1}
\author{Okko Makkonen\thanks{Aalto University}}
\date{February 12, 2023}

\begin{document}

\maketitle

These are some fundamental and elementary theorems within the mathematics done in ancient Greece with proofs in modern language.
    
\section{Problem 1}

\begin{claim}
The square root of $2$ is irrational.
\end{claim}

\begin{proof}
Assume that there exists $a, b \in \Z$ with $\gcd(a, b) = 1$ such that $\frac{a^2}{b^2} = 2$. Then $a^2 = 2b^2$, so $a$ is even. Thus, $a = 2c$, and $2c^2 = b^2$, which means that $b$ is even. This contradicts our assumption that $\gcd(a, b) = 1$, so $\sqrt{2} \notin \Q$.
\end{proof}

\section{Problem 2}

\begin{claim}
There are infinitely many primes.\footnote{This was first shown by Euclid \cite{euclid400BCE}.}
\end{claim}

\begin{proof}
Assume that there are finitely many primes, say $p_1, \dots, p_N$. Consider the number $q = p_1 \cdots p_N + 1$. It is clear that $p_i \ndiv q$, so by the fundamental theorem of arithmetic there must be another prime factor of $q$ that is not on the list $p_1, \dots, p_N$. This contradictics our assumption of having a finite number of primes.
\end{proof}

\bibliographystyle{ieeetr}
\bibliography{bib.bib}

\end{document}